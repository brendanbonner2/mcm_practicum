\section{Conclusions and Recommendations}

The conclusion of the practicum is there is currently no simple and efficient method of verifying the lifecycle of the development of the model. When models are released, either for production, competitions or uploaded to critical systems, they remain a blackbox. While developing a system for verifying the history of the model, it will allow us to both gain an insight into the development process, but also be able to check shared characteristics between machine learning deep learning systems.

\begin{enumerate}
    \item 
    Increasing information stored
    Creating the informaiton blocks at a more granular level. We want to be able to generate the checksum with all the information currently stored, but we would create a non-optimal storage if we included the full set of model weights. There may be a compromise to provide more visibility, while retaining unique characteristics of the model.
    
    \item 
    Introduction of blockchain verification
    Currently using database solutions that can be accessed publicly, and member linewidth
    
    \item 
    Closer integration with Machine Learning providers
    Including push functionality into the training and deployment process could be transparent
    
    \item 
    Streaming of Information
    Streaming, via a node.js server, directly to a web page based on a local "training id" would assist in verifying what is currently being trained.
\end{enumerate}


\subsection{Recommendations}


This section is usually left until the rest of the paper has been written. Conclusions are drawn in the context of the objectives of the project. They should be supported by data and results, and, if possible, compared with theory and data obtained by others in the literature (i.e. related published work). It is a chance to summarise what you have learnt from the project. Remember that human attention spans are extremely short, and the reader will appreciate a good summary, even if you feel that your conclusion is self-evident.

Recommendations are also extremely important, as they provide an opportunity to demonstrate the experience you have gained. The ability to self-assess one's work with a view to suggesting ways things could have been carried out differently is a valuable asset.

Recommendations can also include suggestions to how the work could be expanded or extended. This is often placed in a separate section entitled "Future Work".
The IEEEtran class file is used to format your paper and style the text. All margins, 
column widths, line spaces, and text fonts are prescribed; please do not 
alter them. You may note peculiarities. For example, the head margin
measures proportionately more than is customary. This measurement 
and others are deliberate, using specifications that anticipate your paper 
as one part of the entire proceedings, and not as an independent document. 
Please do not revise any of the current designations.

