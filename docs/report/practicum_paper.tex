%\documentclass[conference]{IEEEtran}
\documentclass[twocolumn,twoside]{IEEEtran}

\IEEEoverridecommandlockouts
% The preceding line is only needed to identify funding in the first footnote. If that is unneeded, please comment it out.
% \usepackage{cite}
\usepackage{amsmath,amssymb,amsfonts}
\usepackage{algorithmic}
\usepackage{graphicx}
\usepackage{tabularx}
\usepackage{textcomp}
\usepackage{xcolor}
\usepackage{listings}

\usepackage[T1]{fontenc}
\usepackage{times}

%\usepackage{biblatex} %Imports biblatex package
%\addbibresource{bibliography.bib} %Import the bibliography file

\usepackage[numbers]{natbib}

\graphicspath{{images/}}


\def\BibTeX{{\rm B\kern-.05em{\sc i\kern-.025em b}\kern-.08em
    T\kern-.1667em\lower.7ex\hbox{E}\kern-.125emX}}

\begin{document}

\lstdefinestyle{mystyle}{
    basicstyle=\ttfamily\footnotesize,
    breakatwhitespace=false,    
    xleftmargin=4pt,     
    breaklines=true,                 
    captionpos=b,                    
    keepspaces=true,                 
    showspaces=false,                
    showstringspaces=false,
    showtabs=false,                  
    tabsize=2,
    keywordstyle=\color{black}\bfseries
}

\lstset{style=mystyle}



\title{
Recording the Lifecycle of Software Models
}

\author{\IEEEauthorblockN{Brendan P. Bonner}\\
\IEEEauthorblockA{\textit{School of Computing} \\
\textit{Dublin City University}\\
Dublin, Ireland \\
brendan.bonner2@mail.dcu.ie}

}

\maketitle

\begin{abstract}
Artificial Intelligence is built on trustworthiness of the systems that provide the intelligence. The foundation of this is built upon our understanding of the role of software models in this process. Similar to measuring social trust in humans, we can only increase our levels of trust by being exposed to the full competence and vulnerabilities of the environment to get to the point in time where trust can be assessed. The pathway for the human brain to grow and evolve is recorded in genealogy, education and experience. In this paper, we investigate feasibility and benefits for replicating this process for machine learning systems. During this process, we create an ecosystem where AI systems can have their entire lifecycle recorded and made available for scrutiny. The framework for a secure chain of provenance of the lifecycle of artificial intelligence systems may be a first step into addressing shortcomings in AI research which potentially could stymie innovation for certain critical systems from being deployed. The outcome of the research cannot coerce trust, although it provides additional insight into comparing aspects of a software models \textit{personality}, without needed to directly examine every neuron. 
\end{abstract}

\begin{IEEEkeywords}
Artificial Intelligence, Provenance, Explainability, Model Analysis\end{IEEEkeywords}

\section{Introduction}
Within the field of machine learning, there is an issue surrounding the lack of transparency of artificial intelligence solutions that are being deployed in commercial and public domains. This has led to initiatives such as the EU developing a white paper on Trustworthy AI \cite{high-level_expert_group_on_ai_ethics_2019}, which has provided an overview of the trust level needed in AI in \cite{ryan_ai_2020}. The most difficult to ascertain is the AI technology itself.

Within the HLEG description, there are a number of references to the \textit{black box} of articial intelligence, where the complexity of a AI Neural Network system can no longer be analysed in a manner similar to statistical systems. Numerous attempts have been made to look inside the layers of the neural network to see what neurons are being activated \cite{kim_interpretability_2018}, and visualising where the attention is focused prior to making a prediction, but so far no provision has been made to establish trust in a system in a method closer to the human environment.

When a person visits a doctor, barrister or any expert, they develop trust because they can access their life experience that will give them the competence to provide the service. We can look up their professional and academic qualifications, review their professional and associative history, and use our udnerstanding that these provide the provenance of trust. It is with this analogy that we will investigate if we can establish a similar provenance of trust for AI systems. The aim is straightforward, can we create a \textit{linkedin} for software models that we can use to evaluate the creation, initial seeding, and then layers of training needed to improve the trust of AI systems. If we can look at the CV of a software model, we should be able to show which base model the system was built on (i.e.  ) and common training (i.e. on \cite{deng_imagenet_2009})

\subsection{Generating a Model Synopsis}
To deliver the model synopsis, a reduction algorithm was applied to each layer in the model, reducing the weights and biases to a pair of unique identifiers containing the \textit{standard deviation} and the \textit{variance}, representing the distribution skew of the data away from a normal distribution. Once this synopsis is applied, we were able to reduce all default keras models from several megabytes to several kb arrays that are comparabile and the basis for generation of a unique signature.

\subsection{Objectives}
The overall objectives of the practicum are as follows:

\begin{itemize}
    \item Discover a method to extract a unique signature from any Deep Neural Network software model.
    \item Create a local and global verification trail for
    \begin{itemize}
        \item A Path of provenance to all ancestor models
        \item A method to evaluate divergence of the child model from the parent
    \end{itemize}
    \item Validation of Provenance via verification and illustration
\end{itemize}

Once the signature is generated, we are able to create a immutable link to a previous model, and store the deviation of both the structure and the weights as a factor. The key element of this is the generation of a lineage for both training and validating that there is a history of the model's evolution.

\subsection{Introduction Simple}
Within the field of machine learning, there is an issue surrounding the lack of transparency of artificial intelligence solutions that are being deployed in commercial and public domains. This has led to initiatives such as the EU developing a white paper on Trustworthy AI \cite{high-level_expert_group_on_ai_ethics_2019}, which has provided an overview of the trust level needed in AI in \cite{ryan_ai_2020}. The most difficult to ascertain is the AI technology itself.

Within the HLEG description, there are a number of references to the \textit{black box} of articial intelligence, where the complexity of a AI Neural Network system can no longer be analysed in a manner similar to statistical systems. Numerous attempts have been made to look inside the layers of the neural network to see what neurons are being activated \cite{kim_interpretability_2018}, and visualising where the attention is focused prior to making a prediction, but so far no provision has been made to establish trust in a system in a method closer to the human environment.

When a person visits a doctor, barrister or any expert, they develop trust because they can access their life experience that will give them the competence to provide the service. We can look up their professional and academic qualifications, review their professional and associative history, and use our udnerstanding that these provide the provenance of trust. It is with this analogy that we will investigate if we can establish a similar provenance of trust for AI systems. The aim is straightforward, can we create a \textit{linkedin} for software models that we can use to evaluate the creation, initial seeding, and then layers of training needed to improve the trust of AI systems. If we can look at the CV of a software model, we should be able to show which base model the system was built on (i.e.  \cite{project_sherpa_httpswwwproject-sherpaeuethics-by-design_2019}) and common training (i.e. on \cite{deng_imagenet_2009})

The topic covers the recording of CNN layer metrics during the training
sequence. The hypothesis is to examine trust mechanisms for explaining AI, and
providing an immutable record of the training sequences prior to a model being
utilised or customised.

The practicum focuses on two approaches to this;
recording and visualisation of the sequence of changes during the training of a
model. During model fitting at configurable intervals, layer characteristics are
measured and recorded, alongside mathematical verification, and stored. This can
be visualised, supporting modern explainable AI measures, to verify model
behaviour according to training, and can be trusted.

The proposal builds upon the attempts to provide additional metrics to support
the establishment of the concept of fairness in AI. In comparison with human
trust, which is established through verification and measurement of academic and
professional credentials, Fairness in AI is an addendum to the model being
examined, with attempts to quantify via measures such as Thiel scores for
certain bias tests. This is further developed by IBMs AI360 Fairness toolkit
paper which show how this is measured and corrected, but not being able to
identify where that behaviour was acquired and why a model is trained in a
certain way to develop these outcomes.

There is currently a gap in identifying if there is a provenance of trust when using deployed AI systems, and if sibling
and inherited models retain these behaviours and if this can be visualised.
While attempts to visualise the internals of decisions in the Simonyon and Bau
papers, these focus on trying to visualise the pathways being activated in a 2
dimensional map. The dissection and visualisation papers provide a key how to
measure and record this aspect of models, and our research will examine if there
is a potential to securely record neural network snapshots.
 
The question being asked is to determine if we develop a novel mechanism and metric for
representing the evolution of knowledge within a neural network during the
training and development stages of creating a model. By analysing the state of
the art in terms of AI fairness and methods of visualising the dissected layers
common networks, the research will establish if creating an immutable chain of
layer delta changes over time will provide an insight into the trustworthiness
of the underlying model.

 The second part of the research will ask if the
information being stored can be visualised based on the elements of explainable
AI to be able to identify which iteration and epoch of the training process is
responsible for the manifestation of the behaviour. Ultimately, the outcome of
the research should ask if providing verifiable additional data on how an
artificial intelligence system evolves over time, if this can provide a
foundation into further research towards comparisons of human trust and
artificial intelligence trust.


\section{Background}

Models in CNNs are the primary reason for opacity of neural networks. Decision
trees and algortihms have the benefit of being easily repeatable outside of the
system. The complexity of even the simplest models makes this unviable. 
The introduction usually describes the background of the project with brief information on general knowledge of the subject. This sets the scene by stating the problem being tackled and what the aims of the project are.

\subsection{The Role of Models in CNNs}
A model in deep neural networks is the basis for the fuzzy nature of a certain outcome based on an input. As the input changes, the impact on weights, balances and activation functions in trained models can provide significant changes in outcomes, whether we a are classifying the image, identifying components or using the outcome as the basis for generating an outcome. In frameworks like Keras and Pytorch, the Model contains roughly the same structure, based on an input layer, a number of hidden layers, and finally a clasification or output layer. As a model is trained, the input layer and all not weighted layers remain the same, but the back propagation function continuously changes the weights and biases across multiple layers with diverse interlinks.
\subsection{Identifying Differences in Models}
Each layer is made up of a structure containing a dictionary of details, plus a multi dimensional array containing the weights for each sublayer that covers all the dimensions, plus a series of biases. While trying to understanding the internals of layers is interesting but relatively futile, being able to break down the structure of the layer into a unique single identifier that is non-repeatable across the smallest changes, but also allows comparison between instances of the layers when trained.
\subsection{Weights and Biases}



\section{Method}
Method should outline how the task/experiment was carried out, including rationale for any decisions made. Details of any equipment and subjects used should be also included. Basically, you should include enough information, so that the reader could duplicate as much of the experimental conditions or design details as possible.



\section{Results and Discussion}

The practicum was able to identify the internal structure of the models used for machine learning applications, and abstract the content to create a single 32 byte indicator, that can be used as an identifier for uniquely identifying the provenance of machine learning models. The two key applications; recording the changes of the model during the training process locally, and publishing the signature to a cloud repository for verification, validation and traceability all resulted in a mechanism for recording the lifecycle of models.

Beyond looking at simple models, we were able to generate signatures for all default Keras models, as well as sample models used in competitive machine learning challenges. In all cases, we were able to generate a \textit{family tree} of the model's history, as well as an \textit{ancestor} function that identifies the original model used as the baseline for published models. Following the process presented in Figure \ref{fig:lifecycleModels}, we are able to store multiple relationships between models as they are created and published. More importantly, we are not able to remove the model from the global repository.

\begin{figure}[!t]
    \centering
    \includegraphics[width=2.5in]{flowchart-4.png}
    \caption{Lifecycle of Software Models}
    \label{fig:lifecycleModels}
\end{figure}

\subsection{Reduction in size for model identification}
The first straightforward result, is being able to produce simplified impressions of models that negate the requirement to store large binary models containing all of the weights, while still providing a utility of being able to determine that a model has changed, how it's structure changes, and which layers have been amended. Below in table \ref{table:memory} is the table of the most common models sizes, which show how much memory is required per iteration, epoch or batch if we wish to retain details of the model. As seen, it shows the high memory cost for some model, which are extracted down to less than one megabyte for the most complex models.

\begin{table}[!ht]
    \centering
    \caption{Memory Usage for Internal Model}
    \label{table:memory}
    \setlength\tabcolsep{0pt} % make LaTeX figure out intercolumn spacing
    \begin{tabular}{@{} p{2cm} p{2cm} p{2cm} p{1cm}  @{}}
        \hline
        Model & Size & Impression Size & Time\\
        \hline
        VGG16       & 528MB & 38.2kb & 4.11s \\
        DenseNet121	& 33MB & 693.8kb & 0.37s \\
        DenseNet169	& 57MB & 952.1kb & 0.53s \\
        Xception	& 88MB & 219.4kb & 0.25s \\
        ResNet50	& 98MB & 292.0kb & 0.41s \\
        ResNet50V2	& 98MB & 311.3kb & 0.42s \\
        MobileNet	& 16MB & 149.3kb & 0.17s \\
        MobileNetV2	& 14MB & 248.5kb & 0.20s \\
        \hline
    \end{tabular}
\end{table}

    Overview of Layers stored in Database
    Comparison of Model Visualisation and Summary

\subsection{Identification of Model History}

One of the most powerful elements for AI practitioners, is the ability to generate a timeline of releases based on when a model is released. In the example shown in figure \ref{fig:lifecycle_in real}, The initial model is shown preserved in both the local and remote repositories. At the end of the model training process, each individual iteration is stored internally with a parent<=>child relationship as part to training process. Once a model is ready for internal testing, it can be baselined, updating the model to point to the original base model. This is shown in the iterations in light blue below. When a model is ready for public release, then the model is baselined.

\begin{figure}[!t]
    \centering
    \includegraphics[width=2.5in]{lifecycle.png}
    \caption{Lifecycle of Models in Development Path}
    \label{fig:lifecycle_in real}
\end{figure}

The true power of this is being able to use the system to verify and baseline systems, so they are compliant and can be audited in line emerging standard in AI, such as the EU Artificial Intelligence Act, inspired by the work of the AI HLEG \cite{high-level_expert_group_on_ai_ethics_2019}. Every subsequent model released will be added to the chain, and at anytime the model itself can be used to generate a signature while shows the history.

    Showing the history of the models either locally or remotely
    Identifying the differences between models
    Family Tree of Models

\subsection{Difference between published Models}
    Obtaining an overview of parameter changes of official models

\subsection{Performance}








\section{Conclusions and Recommendations}

The conclusion of the practicum is there is currently no simple and efficient  method of verifying the lifecycle of the development of the model. When models are released, either for production, competitions or uploaded to critical systems, they remain a blackbox. While developing a system for verifying the history of the model, it will allow us to both gain an insight into the development process, but also be able to check shared charactaristics between machine learning deep learning systems.


\subsection{Recommendations}

This section is usually left until the rest of the paper has been written. Conclusions are drawn in the context of the objectives of the project. They should be supported by data and results, and, if possible, compared with theory and data obtained by others in the literature (i.e. related published work). It is a chance to summarise what you have learnt from the project. Remember that human attention spans are extremely short, and the reader will appreciate a good summary, even if you feel that your conclusion is self-evident.

Recommendations are also extremely important, as they provide an opportunity to demonstrate the experience you have gained. The ability to self-assess one's work with a view to suggesting ways things could have been carried out differently is a valuable asset.

Recommendations can also include suggestions to how the work could be expanded or extended. This is often placed in a separate section entitled "Future Work".
The IEEEtran class file is used to format your paper and style the text. All margins, 
column widths, line spaces, and text fonts are prescribed; please do not 
alter them. You may note peculiarities. For example, the head margin
measures proportionately more than is customary. This measurement 
and others are deliberate, using specifications that anticipate your paper 
as one part of the entire proceedings, and not as an independent document. 
Please do not revise any of the current designations.



\bibliographystyle{IEEEtran}
\bibliography{bibliography}


\end{document}

